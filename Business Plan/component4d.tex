\documentclass[10pt, a4paper]{article}
\usepackage{fullpage}

\begin{document}
	
\setlength{\parskip}{6pt}
\setlength{\parindent}{0pt}
% describe the strategy in which you will bring that product or service to market, including the following:
% What is your approach to financing?
\section*{Financing} \label{financing}
Our business benefits from both very low setup costs and the potential for incremental automation to scale. The only fixed costs before any cards can be shipped will be for registering the company, which costs £12, and a domain name, which costs £10. The costs incurred per card will depend on the scale of the business---as we form partnerships and buy items in bulk, our costs per card will decrease. Table \ref{table:costpercard} shows these costs at both small and large scale. Moving from small to large scale will be a different process for each line item, indicated in the Notes column. The item with the biggest impact, the card, will be the first priority since seller's partner programs cut purchasing costs dramatically.

Other fixed setup costs such as purchasing a printer are not counted because this equipment is already owned so doesn't need to be purchased by the company. A later purchase of a better printer may be necessary, but will not count as a setup cost.

Ongoing costs will also include web hosting, email provision and server costs as the service grows. However, the cost of these will scale slowly with the size of the business, as shown in Table \ref{table:costpermonth}. Services such as rented office space will not be required until the company reaches a very large size, due to the low per-customer work required. This also means the company can grow large before needing to pay employee wages, with the small amount of manual work until then performed on a voluntary basis by the founders.

In order to not have to work directly with bank details, we will use a payment processor such as Stripe. Payment processor fees are typically around 1\% with a 20p minimum. For us, this means it will be worthwhile to bill multiple cards at the same time---for instance, a person subscribing to three cards annually would be billed for the cost of all three, once per year. This reduces the per-card processor fee to 7p.

In order to be profitable with these costs, a basic card subscription would cost around £5, excluding tax. As the increasing scale causes costs to decline, the excess can be used to hire a part-time employee and rent office space. In order to reduce costs quickly, it may be necessary to register for programs such as Etsy's partner program before starting to sell cards.

As the company will not have large setup costs, no external funding is required---the business will become profitable after the first few sales. However, funding may be required to accelerate growth.

\appendix
\section*{Financial Data}
\begin{table}[]
	\centering
	\caption{Cost per card at small vs large scale}
	\label{table:costpercard}
	\begin{tabular}{llll}
		Item                    & Cost at small scale & Cost at large scale & Notes                      \\ \hline
		Card with envelope      & £3.00               & £1.50               & Partner program       	 \\
		Envelope (to customer)  & £0.05               & £0.02               & Bulk buying                \\
		Cover letter            & £0.05               & £0.02               & Better printer             \\
		Address label           & £0.01               & £-                  & Print directly to envelope \\
		Delivery (to customer)  & £0.56               & £0.39               & Franking                   \\
		Delivery (to recipient) & £0.65               & £0.65               & First-class stamp included \\
		Printing of address etc & £0.02               & £0.01               & Better printer             \\
		Payment processor fees  & £0.25               & £0.07               & Reduced fees, batching     \\ \hline
		Total per card          & £4.59               & £2.66               &                           
	\end{tabular}
\end{table}

\begin{table}[]
	\centering
	\caption{Cost per month at small vs medium scale}
	\label{table:costpermonth}
	\begin{tabular}{llll}
		Item            & Cost at small scale & Cost at medium scale & Notes               \\ \hline
		Web hosting     & £5.00               & £25.00               &                     \\
		Email           & £-                  & £-                   & Included in hosting \\
		Domain name     & £0.83               & £0.83                &                     \\
		Backend         & £-                  & £5.00                &                     \\ \hline
		Total per month & £5.83               & £30.83               &                    
	\end{tabular}
\end{table}
\end{document}
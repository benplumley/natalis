\subsection{William Kennerley}\label{subsec:williamKennerley}
My most important individual contribution to the coursework was in the first section, \textit{Our Product}. In this section I introduced the problem we are trying to solve, the consumer needs, described the alternatives which we considered, helped to write the value proposition and discussed the risks we recognised as part of the risk analysis. In addition to this, I wrote the cover sheet elevator pitch,I gave guidance to the team based on my previous experience writing a business plan and made suggestions across the document when proof reading.
There were two major problem areas which I faced:
\begin{enumerate}
  \item Determining consumer needs
  \item Narrowing down alternatives to construct a focused value proposition
\end{enumerate}
In response to the challenge of determining consumer needs, I read the academic source material for the following key theories on human motivation and consumer values:
\begin{itemize}
  \item Maslow's Hierarchy of Needs
  \item McClelland's Human Motivation Theory
  \item Herzberg's Two Factor Theory
  \item Deci and Ryan's Self-Determination Theory
  \item Sheth, Newman and Gross' Theory of Consumption Values
\end{itemize}
I had to choose which of these theories was most appropriate to determine the needs of our customers. I first ruled out using McClelland's Human Motivation Theory and Herzberg's Two Factor theory as they focus on motivation within organisations and they do not address why people buy products. Likewise, I ruled Deci and Ryan's Self-Determination Theory as the most influential conclusion of this theory, intrinsic motivation, could be implied when discussing the social value \citep{sheth1991we} which greetings cards provide.
I chose to use Maslow's Hierarchy of Needs for its fundamental importance in psychology and its usefulness in identifying the basic needs which our product would fulfil. Recognizing the consumer needs was essential, as it let us determine if any competitors already met all the needs of our target market. If a competitor did match all the needs of our target market, we would need to restructure our VP or pivot.
I also chose to use the Theory of Consumption Values by Sheth, Newman and Gross as it could be used to accurately pinpoint why consumers would want to buy our product and not our competitors. By identifying why consumers choose to purchase products and more specifically, greetings cards, we could design Natalis to be the most appealing player in the market.
After determining the needs of the consumer and their consumption values, narrowing down the features (alternatives) of our product to form a value proposition was straightforward. To do this, I listed all the proposed features along with the consumer need or value they fulfilled. Where a need or value was fulfilled by two features of the product, one of the features was planned for removal. To confirm the choice of which features to remove, we conducted primary research with potential users. I contributed to the primary research by holding some interviews. Through this process, I could help to construct a focused value proposition.
\clearpage
\subsection{Ben Plumley}\label{subsec:benPlumley}
 The most important individual contribution I made to the coursework was the research and writeup of our competitors and suppliers. This made up around half of the Industry Analysis section, and part of the Risk Analysis section.
This involved researching the business models and supply chains of the following companies:
\begin{itemize}
	\item Hallmark
	\item Clinton
	\item Card Factory
	\item Cards Galore
	\item Sainsbury's
	\item Tesco
	\item Archway Cards
	\item numerous independent shops
\end{itemize}
Problems encountered while researching this section included the fact that shops are unlikely to provide information about their supply chain, possibly to maintain a competitive advantage. However, by analysing the portfolios of larger suppliers, I was able to infer which shops were supplied by which companies.
This allowed me to make like-with-like comparisons between shops using the same suppliers, and will also help us to cut out extraneous middlemen and resellers when finding our own suppliers.
Once I had an accurate idea of the current competitive landscape, I was able to look further into the portfolios and strengths of each supplier to determine which would be appropriate for us to develop a relationship with. This also ties strongly into the latter half of the Industry Analysis section.
My findings showed that while the businesses I'd analysed could supply cards to us, they would only be able to supply the lowest-end cards we offer, due to the way the target markets overlap. This informed research into the hand-made and artisan markets which would make up the remainder of our suppliers.
Additionally, I proofread and offered suggestions and improvements on all other sections of the document.
\clearpage
\subsection{Max Sandberg}\label{subsec:maxSandberg}
My most important role in the creation of the business plan was having joint responsibility for the Industry Analysis section. Once we had all agreed on the business we wanted to make and how it would work, we delegated different sections of the business plan to different members of the group. As there are 6 group members, we assigned 2 people to each of the first 3 sections, and agreed that each pair would also write the part of section 4 that was relevant to their section. The two of us then formed a list of all the companies we wanted to discuss in this section, and split this list between us. I was left with the following list of companies to research and consider our relationship with:
\begin{itemize}
	\item Moonpig
	\item Funky Pigeon
	\item Etsy
	\item Redbubble
	\item M\&S
	\item John Lewis
	\item Facebook
	\item Whatsapp
\end{itemize}
Once we had all written our sections, we compiled them into a single document. I feel that this approach was a fair and effective way of splitting the workload across the group. I also proofread and made suggestions on the rest of the document, then implemented the suggestions others had made for the parts that I wrote. I believe this lead to a significant improvement in the overall quality of the business plan.
One particular development in our business strategy came about as part of my research into e-commerce businesses such as Etsy and Redbubble. Our understanding of these websites previously came only from our previous experience using these websites as consumers. This meant that when we discussed using sellers from these websites as our suppliers, we imagined this sales process to be much like the process of buying cards from these sellers as an individual consumer. It was only when looking into their terms of service that I discovered these sites have an entirely different scheme, designed for businesses to buy from the sellers at wholesale prices.
This was an interesting development, particularly due to the wholesale prices of these cards being at least half that of the normal retail prices. This is a significant saving over the prices we had expected to be paying, and therefore a big improvement to the potential profitability of the business. I discussed these findings with the team, and we decided that buying our supplies through one of these wholesale schemes would be a good decision. I then went back and made the appropriate changes to the business plan to reflect this change.
\clearpage
\subsection{Oliver Gray}\label{subsec:oliverGray}
My primary focus was on market validation. I was involved in the creation and refinement of the original interview question set, and conducted six interviews with potential users and recipients. In addition to this I carried out the secondary market research. As result, it followed that I would work on the ``Market Validation'' section of the report, in particular the ``Importation Questions'' and ``Secondary Research'' subsections.
A key decision was on what questions to ask the interview participants. For this I took the ideas that were found during brainstorming and from the consumer needs analysis, and conducted some interviews using those ideas. This process strengthened some of the original ideas we had, and also allowed us to discard some others that we found were unpopular with the initial interviewees.
In addition to this, my role was to determine the current state of the market, so that we could show there was space for us to operate within it. This would clearly require research. I faced few problems finding useful secondary information as there are many greetings card associations and market research reports with statistics that are easily available.
The only barrier to the research was that only a summary of the report was available without paying a substantial amount of money (typically \textsterling750-2000). However, I think that a summary of the information is acceptable at this stage as we are only interested in finding out if there is a market for our idea, as opposed to optimising our performance within that market.
\clearpage
\subsection{Caroline Moir}\label{subsec:carolineMoir}
My main contributions to the coursework were interviewing people to acquire information for our idea, analysing their feedback and exploring which factors differentiate our idea from our competitors. In the final business plan, I also explored the roles of each of the team members, and how the areas in which we are each skilled in help to produce an effective help. Ollie made the questions for our first interview and then him, me and Will each interviewed people with them. I then analysed this feedback, created a second interview template, and used it to further interview some more people. I also assessed our main competitors, and explored the defining features of our idea which would differentiate us from them.
Therefore the part of the coursework most affected by my contributions was the market validation section in component 3, more specifically parts C and D of it, and also component 4b.
One of the problems I faced was the struggle of finding a wide range of people to interview. I wanted to ensure that I got a broad range of opinions and did not only interview one group of people, i.e.\ students, as these were the group of people who were easiest for me to talk to. Also, another problem I experienced when conducting interviews was trying to ensure that we weren't affected by the ``mom test'' and social biases. The people I was interviewing may have been enthusiastic for any system I was proposing because they wanted to give me nice feedback, not because they actually would use the system.
I researched how to avoid our interviews being affected by the ``mom test'', and found that it was best to avoid asking questions specifically about our business idea and whether the interviewee would use it or not. Instead, a better idea was to ask them more generally about a scenario they have experienced and ask what problems they might have encountered during it.
To ensure that I was receiving varied opinions from different groups of people, I interviewed some people who I used to work with on placement, along with fellow students and family members. In order to not be a victim of the ``mom test'', I made interview questions such as ``have you ever experienced any problems when doing this'' and ``would you prefer x or y'' rather than ``would you use x''.
\clearpage
\subsection{Darien Opperman}\label{subsec:darienOpperman}
The contributions I made that had the largest impact on the project were my contributions to the "Our Product" section. Specifically, the \textit{Value Proposition} and \textit{Meeting Customer Needs} sub-sections. To complete these sections, a concrete, unified concept of the service had to be established within the group. Having a clear, well defined idea of the value proposition within the group allows for all other sections to be consistent.
As part of my effort to create a unified concept, I designed a series of logos, and edited them based on group feedback, going through multiple internal feedback stages.
The difficulties in clarifying the value proposition were due to subtle differences in perceptions and understanding. If one group member says the believe our business should be "classy", how do they interpret that word, and what implications does it have for our service? Lots of discussion within the group, and usage of other businesses as examples, were used to create an idea that I believe is now consistent within the group.
The differences in perception were highlighted in responses to logo design. Opinions on what an appropriate logo would look like varied within the group, with many members submitting suggestions and examples, some varying wildly, beyond simple differences in aesthetic taste. As a result, I researched logos, and the perception that consumers can generate based on a logo. This included research into existing popular logos, but also more abstract concepts such as emotional response to different colours.
My initial logo drafts relied too heavily on symbolic references to gift giving, sentiment, and warm social circle type feelings. Group response was negative due to this, and due to similarities between the imagery and charity company branding. Researching further, I found that symbolic references are not always required, and customers can create the sentimental links personally with any logo. Based on this, and more stages of group feedback, more logo drafts were devised, relying less on heavy symbolic meaning, and more on clean minimal branding that clearly indicated that greetings cards are the focus of the service.
On top of the above, my input also included some proofreading and small suggestions for other areas of the document.
\end{appendices}
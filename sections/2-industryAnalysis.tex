\subsection*{Direct Competition}
% BEN
Natalis will coexist with a few well established industry incumbents, along with a number of younger and faster-growing new entrants. From a functional standpoint, Natalis will work alongside traditional card companies. For instance, Hallmark is a well established traditional card designer and distributor. Historically, the market was dominated by companies in this category. As our business model will involve convincing customers of the convenience and peace of mind advantages that we offer, our primary sources of competition will be brick-and-mortar card shops or supermarkets, which almost certainly carry some Hallmark cards.

However, given that Hallmark is in the business of mass-producing cards, there is also scope for them to be a supplier of a subset of the cards we offer to our customers as part of our service. This scope is limited by the fact that few Hallmark cards satisfy the standards for taste and quality which will be imposed on our selection of cards. This is true of many consumer-facing card shops, meaning that while we may buy some of their cards to offer as part of our service, the majority of the cards they carry will not satisfy the tastes of our customers. Thus, these businesses may become our suppliers, but only of a small portion of the cards we offer.

A number of other companies fall into this same category, such as Clinton. In general, the market for this type of dedicated high-street card shop is declining in favour of the alternatives described below. This is due to the inconvenience of these dedicated shops compared to grocery shops (where consumers are likely to find themselves regularly), and lower profit margins compared to companies such as Card Factory.

Some companies' offerings appeal to the same segment of the market as Hallmark, with the exception of being vertically integrated with the storefront distribution channels. For instance, Card Factory brand cards are only sold in Card Factory stores. From a competitive standpoint, this will pose a similar challenge as mentioned above, however their vertical integration will prevent them from being a supplier in any official sense. The same caveat as with Hallmark also applies, because the vast majority of cards carried by Card Factory do not satisfy the design criteria required to appeal to our target audience. Cards Galore is another example of a company in this category.

One of the most common places for people to buy cards is in a section or aisle of a general-purpose shop, for instance Sainsbury's or Tesco. These are typically supplied by a combination of businesses in the first category (Hallmark, etc.), and also tend to cater to the lower end of the market, often only offering one or two high-end cards which would match our offerings. Because of this, these companies pose little competition, and may be used as our supplier for a subset of the cards we offer.

Smaller general-purpose shops (for instance, independent grocers or caf\'es) often use a third party supplier such as Archway Cards. This supplier is responsible for choosing, delivering and restocking the card selection of the shop. A supplier such as Archway Cards could be a very useful partner, since they will be capable of adapting to our requests for high-end cards. Since they do not distribute directly to consumers, they are not a competitor, though the grocery shops they supply will compete in the same category as larger general-purpose shops such as Sainsbury's.

% MAX
One of the primary selling points of Natalis is convenience. For customers who prioritise convenience when deciding where to buy cards, some of our largest competitors will be online card retailers such as Moonpig and Funky Pigeon. These businesses allow the user to browse their selection of cards from their home via a website. The user can then choose a message to put in the card, and the company will deliver it directly to the chosen recipient.

Moonpig and Funky Pigeon's card prices are generally towards the lower end of the market. This is possibly done to create a more reasonable overall cost, as customers must account for the price of shipping as well. Because these companies are very well established in the market, competing with them directly would be very difficult. By targeting the higher end of the market, which isn't catered to by existing incumbents, we aim to minimise the direct competition we have with these companies. However our relationship with them is still a competitive one, as there remains some level of overlap between our target markets.

We must also consider companies like Etsy and Redbubble. These are e-commerce websites focused on handmade or vintage items and supplies, including high quality handmade cards. Most of our high quality cards will be bought from one of these websites, through their retailer schemes which allow businesses to buy wholesale from the sellers at a reduced price. Our relationship here isn't with the e-commerce platforms, but instead with the sellers who utilise them. We hope to build a network of trusted card-makers through these websites to act as our suppliers. In the long term we hope to bypass the e-commerce platforms, and deal directly with the card manufacturers we have established relationships with. However, our relationship with these sellers is interesting, as they act as both suppliers and competitors. There is nothing to stop our customers from buying their cards directly from the sellers via these sites. For customers who are aware they have this option, we hope they will be swayed by the additional convenience our service provides.

\subsection*{Indirect Competition}
In order to find our indirect competition, we must consider what other actions people might consider when preparing for the birthday (or other event) of a friend or family member. If the customer's relationship with the person is close, they may consider a gift, either instead of or alongside a card. A common place to buy gifts would be a home product retailer such as M\&S or John Lewis. This appeals to those who want to do something particularly thoughtful for their friend or family member, and are willing to spend a bit more to do so. The overlap between this audience and our own target market is significant; as such, our relationship with these businesses may prove to be more competitive than the aforementioned alternative card retailers. These consumers should be targeted with marketing which emphasises the convenience of Natalis without compromising quality.

Another more modern form of indirect competition is simply sending a message to the recipient via social media or a messaging application, such as Facebook or Whatsapp. While this clearly doesn't carry the same sentimental value as a gift or card, it still shows some level of thought on the part of the sender. The clear advantage of this approach is convenience for the sender, as no preparation is required. By highlighting the unimpressed reaction of a recipient who received an online message, and comparing to the excited reaction of a recipient who received a card from Natalis, we can illustrate to this consumer group that Natalis can provide joy to their recipient while still requiring little effort and preparation from the user.
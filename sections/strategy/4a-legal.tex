As an online retailer, there are laws we must abide by regarding what information we make clear to customers on our website (e.g. terms and conditions, contact details). These will be thoroughly investigated as part of the website design stage to ensure compliance. We must also consider data protection laws, because is inevitable that we will require some personal information from our customers. Payments will be processed using a third party payment provider (see section~\ref{financing} for details), so we will not need to store any payment details. However, we will need to store personal information such as names and addresses. This means we must adhere to data protection regulations, as infringement of these would leave us liable to fines from the Information Commissioner's Office (ICO).

To ensure compliance with data protection regulations, a data protection impact assessment (DPIA) should be carried out to identify potential security concerns with how data is stored. This is mandatory for compliance with the EU General Data Protectection Regulation (GDPR), which will be enforced from 25th May 2018. The ICO also recommend tasking one member of the team with responsibility for day-to-day security measures. This includes responding to data protection requests within 40 days, and periodic checks to ensure the organisation's security measures remain appropriate and up to date. An example of a security measure that should be taken is encryption of user data stored in SQL databases. Furthermore, under data protection law, we will need to notify the ICO of how our organisation handles personal data about our customers. 

The business structure we have opted to use is a limited liability company (LLC). Unlike a partnership, this means the company is a separate legal entity to the company directors. This reduces our exposure to financial risk, as if the company were to go under (for example, if our data protection measures were insufficient and we were made to pay a fine we couldn't afford), the financial burden lies with the company rather than us as individuals. However, being an LLC comes with significantly more administrative and regulatory demands than a partnership. To become an LLC, we will need to register with the Companies House. As an LLC we will need to keep company records of our transactions and financial agreements, and file our accounts with the Companies House and our company tax return with HMRC. These accounting tasks will be assigned to a member of our team who will be responsible for their completion.

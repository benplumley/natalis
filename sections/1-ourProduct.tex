\subsection*{The Problem}
Consumers across the UK send and receive greetings cards to demonstrate their love and respect for their friends and family. Consumers engage in this practice because of their needs for social belonging and esteem. In particular, greetings cards are chosen to fulfil these needs as they bear substantial conditional and emotional value; two values which consumers require a product provide to fulfil their purpose in demonstrating love and respect. However, in the current market, the choice available to the consumer is tightly constrained by the functional and social needs that they have of the product, limiting them to just a few options.

\subsection*{Consumer Needs}
Natalis aims to fulfil six needs based on fundamental works by \citet{maslow1943theory} and \citet{sheth1991we}.
\begin{enumerate}
    \item Social belonging rests in the centre of Maslow's hierarchy. It theorises that all humans have a desire to be part of social groups of all sizes, and have an innate need to love and feel loved by others. Since their invention, greetings cards have been a popular method to demonstrate one's feelings for another person on meaningful occasions, such as birthdays, religious holidays and family milestones. Through the medium of greetings cards, the sender can demonstrate their love for another person, and in return, the recipient will feel loved upon reading it: both sides of the social belonging need are fulfilled. It is ideal for the sender if the demonstration of their love is reciprocated by the recipient; reciprocity may come in the form of a returned greetings card or another medium.
    \item Emotional values are related to the feeling which the consumer will receive from the product. As greetings cards are supposed to be given as a demonstration of love for another person, the full emotional value of the card is only realised once the card has been received by the recipient. The phenomenon of reciprocated love, described above in the social belonging need, exemplifies the emotional value which greetings cards possess.
    \item Esteem is described by Maslow \citep{maslow1943theory} as the need for a high-evaluation of one's self. Typically, this need is fulfilled through self-esteem, respect for others or respect by others. By sending a greetings card, the sender is demonstrating their respect for another person: whether it is their birthday or a religious holiday, the sender is showing that a significant day for the recipient is of high importance to them too. Likewise, when the recipient opens the card, they are likely to feel a boost in their self-esteem as a result of being respected by the sender.
    \item Functional values are a traditional driver of consumer choice \citep{sheth1991we}. The functional values in the chosen problem space, described above, include: the choice of designs of greetings cards available for purchase, the price of the cards and the time it will take the consumer to complete their customer journey \citep{edelman2015competing}. The customer journey of Natalis starts by visiting the store, followed by writing the card, then visiting the post office and finally getting the card in the hands of the recipient. A product with high functional value, such as a card with a more attractive design or a card from a shop which is closer, is more likely to be purchased by the consumer.
    \item Social values refer to the worth gained from a purchase that leads to a sense of belonging to a community \citep{sheth1991we}. While there is no specific social group or tribe \citep{canniford2011manage} associated with greetings cards, the quality and style of the card, as well as the message that is written on the card, can convey a lot about the social standing of the consumer and the tribes with which they associate. For example, a consumer wishing for the recipient to perceive them as higher social class may purchase a card with a classic design and fill it with a lengthy hand-written message, whereas a millennial wanting to portray their bohemian image might opt for a handmade card which represents their style and interests. Consumers can gain value through the images they present of themselves in the greetings cards they purchase.
    \item Conditional values \citep{sheth1991we} are temporary values which are related to a specific set of circumstances associated with the object the consumer is looking to purchase. By their nature, greetings cards have a strong conditional value as they focus on commemorating a single event. As the event draws closer, the conditional value of the card increases along with the consumers need to purchase a product which fulfils their conditional values. Additionally, the more unique the event, the higher the conditional value of the greetings card. For example, a card for a once in a lifetime event such as a wedding, has higher conditional value than a birthday card which will be bought once a year.
\end{enumerate}
\subsection*{Value Proposition}
Natalis is a web-based subscription service for greetings cards.

Customers enter significant dates relating to each contact, such as birthdays and anniversaries, and are offered a selection of appropriate greetings cards close to the entered dates. Their card of choice will be mailed to them along with a stamped envelope addressed to their recipient. The customer simply writes the inside of the card and posts it when convenient in their nearest postbox. The customer is spared the hassle of visiting their high-street card shop and post office, allowing them to focus on the sentiment.

Users interact with Natalis through our website. Creating an account is required so we can store their contacts' information. Upon account creation, users are prompted to add a contact; the information required for their contact is the important date and event type, so appropriate card options can be offered at a suitable time, and their address, so the stamped envelope can be prepared. Each contact will also have a subscription type: repeating or single. Once the contact is created, they can leave the rest to us.

Two weeks before the chosen date, the customer is emailed a selection of cards which we think they might like, along with a link to the website where they can view the full range of designs. The selected design is then packaged with a stamped envelope, printed with their recipient's address and dispatched to the customer as quickly as possible. Upon receipt of the card, the customer can take the time to fill out a meaningful message inside the card, place it inside the included envelope and post it at their convenience.

There are many mass-produced, generic cards available on the high-street and online. The aim of Natalis is to provide a similar accessibility level for high quality, premium cards. On top of being able to access quality greetings cards, customers also gain the convenience of having the choice of cards emailed to them on a set-and-forget schedule, making the disaster situation of forgetting a birthday or anniversary less of a concern for the customer.

\subsubsection*{Meeting the Customer Needs}
The value proposition of Natalis fulfils the six needs described above.

The need for social belonging is met when customers reach the enjoyment phase of their customer journey \citep{edelman2015competing}: once the intended recipient has received their card and the display of affection within. Natalis provides a mechanism which allows greetings cards to be sent more easily, facilitating customers in fulfilling their social belonging need.

Natalis fulfils the customer's emotional values by offering greetings cards which serve as a symbol of sentiment from the sender. This helps the customer feel accomplished knowing that they have surpasssed the expectations of their recipient. This positive feeling is also mirrored by the recipient of the card, who feels desired and loved, understanding that the sender put effort into them.

The need for esteem is fulfilled through the customer's display of love and respect for the recipient, which is embodied by the greetings cards distributed by Natalis. For the recipient, the emotional experience of feeling loved and valued helps to build their esteem. If consumers forget important dates, they risk their self-esteem. Natalis mitigates this risk to self-esteem through email reminders which are triggered when important dates draw closer.

Natalis meets the functional values of the consumer through two channels. Firstly, the total time taken to complete their customer journey is shortened; most respondents in our market validation surveys visit high-street card shops to buy greetings cards. Not only does Natalis offer a selection of cards over the internet, removing the time taken to travel to high-street shop, it also provides a pre-selected, recommended list of cards to the customer in the email notification they receive, which will reduce the time taken to choose a card. additionally, the included stamped envelope eliminates the time that would be spent queueing at the post office to send their card. Secondly, Natalis will offer a wide variety of high-quality card designs which can't be found in high-street shops; the wide range of unique designs adds to the functional value of the cards.

Similarly, the wide range of card designs can fulfil the social values of the customer. This is because, the customer can pick a card design which they believe fits with their social group, tribe, and self image, while still being appropriate for the recipient.

The timing of the reminder email is key to providing the customer with high conditional value. When the customer sets up their account and adds in the details of their contacts', it is likely that the dates associated with that contact are in far in the future; a greetings card bought at this time would have low conditional value. By sending the customer a choice of greetings cards by email two weeks before the date the card should be received, the conditional value of the greetings card has increased significantly. As a result of this, customers will associate high conditional value with Natalis and become loyal advocates of the service.

\subsection*{Alternatives}
\subsubsection*{Algorithmic Card Selection}
We proposed the idea of integrating an algorithm with Natalis that would aim to select the most appropriate greetings card for the customer's recipient. The algorithm would require customers to link their social networking information with Natalis, allowing us to view their contacts' public profile including their interests, personality and social networking behaviour. By comparing the recipients profile with statistical models trained on data collected from other customers, the algorithm would be able to select the perfect greetings card for the recipient. This concept is already common place in other industries such as advertising and retail.

We discarded this idea for several reasons:
\begin{enumerate}
    \item Social Values: Natalis aims to meet the social values of the customer. By taking away the customer's choice of card, they lose the ability to define themselves by their choice of card which can reduce their sense of belonging to their community.
    \item Emotional Values: Algorithmic selection removes some of the thought process of the customer. While the emotional value of the card will still be realised upon receipt, the sender will have not put as much time into the greetings card process. This reduces the emotional value of the card, and thus the fulfillment of the customer's needs.
    \item Market Validation: Our Market validation revealed to us that most consumers care more about the sentiment of the card and the message inside it, rather then the quality or cost.
    \item Lack of Data: For the algorithm to work correctly, the system must understand which cards were a considered a success by the sender and which were a failure. This would require training data acquired through the use of the product.
\end{enumerate}

The combination of these factors lead us to decide that the integration of a card selection algorithm is unsuitable for Natalis at this time.

\subsubsection*{Gifts}
We proposed the idea of offering gifts with each greetings card sent. Gifts are an excellent companion to a greetings card as they fulfil all the same needs and values of our target consumer. Like the greetings cards, users would be sent a reminder to choose from a selection of gifts one week before the specified card arrival date.

This idea was discarded for a number of reasons:
\begin{enumerate}
    \item Value chain complexity: By adding in additional suppliers, our value proposition becomes more complex. To get the business up and running swiftly, we decided that a simplified value proposition will still fulfil all the needs of the customers while allowing an MVP to be built.
    \item Focus: Gifts do not fulfil any additional needs or values of the consumer compared to greetings cards but do have the potential to fulfil them to a greater extent or in different ways. By offering only cards, we can focus on our core value proposition, and on serving the needs of a small target market in order to establish a core customer base before expanding.
\end{enumerate}

\subsubsection*{Handwriting Service}
We proposed a number of options to write the inside of the card for the customer to provide a more automated service:

\begin{itemize}
    \item AxiDraw Machine.
    \item Upload and print a scanned handwritten message.
    \item Print a message using a unique, personalised handwriting font based on a writing sample.
    \item A handwriting specialist to duplicate a handwritten message (alternative to printing).
\end{itemize}

While these options would fulfil the functional needs for the consumer by saving them time, the potential users we interviewed as part of market validation suggested that printed options would show a lack of effort on the part of the sender. This would result in a reduced sense of emotional value gained from the card, and reduced fulfillment in social belonging and esteem needs in both the sender and recipient.